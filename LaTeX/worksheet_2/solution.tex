% header
\documentclass[10pt,a4paper]{article}

\usepackage[utf8]{inputenc}
\usepackage{hyperref}
\usepackage{amssymb}
\usepackage{amsmath}
\usepackage{listings}
\usepackage{graphicx}

% the document
\begin{document}

\title{Worksheet $2$\\
\small{Practical Lab Numerical Computing}}
\author{Andrii Lischishin \and Lars Schleithoff \and Hendrik Kleikamp}
\date{\today}
\maketitle

\section*{Task 1}

\section*{Task 2}

\section*{Task 3}
To prove that
\[
\mathbb{E}[V_{call}(S_T,0)]=S(0)\exp(\mu T)\Phi(\sigma\sqrt{T}-\chi)-K\Phi(-\chi)
\]
we use that by change-of-variable with $t:=-t$ we get
\begin{eqnarray}
&&\mathbb{E}[V_{call}(S_T,0)] \notag\\
&=&\frac{1}{\sqrt{2\pi}}\int_{\chi}^{\infty}\left(S(0)\exp\left((\mu-\frac{1}{2}\sigma^2)\cdot T+\sigma\sqrt{T}s\right)-K\right)\exp\left(-\frac{s^2}{2}\right)ds \notag\\
&=&\frac{1}{2\pi}\int_{\chi}^{\infty}\left(S(0)\exp\left((\mu-\frac{1}{2}\sigma^2)\cdot T+\sigma\sqrt{T}s\right)\right)\exp\left(-\frac{s^2}{2}\right)ds-K\frac{1}{2\pi}\int_{\chi}^{\infty}\exp\left(-\frac{t^2}{2}\right)dt \notag\\
&=&\Psi-K\frac{1}{2\pi}\int_{-\infty}^{-\chi}\exp\left(-\frac{t^2}{2}\right)dt \notag\\
&=&\Psi-K\Phi(-\chi) \notag
\end{eqnarray}
with
\[
\Psi:=\frac{1}{2\pi}\int_{\chi}^{\infty}\left(S(0)\exp\left((\mu-\frac{1}{2}\sigma^2)\cdot T+\sigma\sqrt{T}s\right)\right)\exp\left(-\frac{s^2}{2}\right)ds.
\]
Now we prove that
\[
\Psi=S(0)\exp(\mu T)\Phi(\sigma\sqrt{T}-\chi).
\]
Again, we use a change-of-variable $z:=t+\sigma\sqrt{T}$ and get
\begin{eqnarray}
&&\frac{1}{\sqrt{2\pi}}\exp\left(\frac{\sigma^2}{2}T\right)\int\limits_{-\infty}^{\sigma\sqrt{T}-\chi}\exp\left(-\frac{t^2}{2}\right)dt \notag\\
&=&\frac{1}{\sqrt{2\pi}}\exp\left(\frac{\sigma^2}{2}T\right)\int\limits_{\chi-\sigma\sqrt{T}}^{\infty}\exp\left(-\frac{t^2}{2}\right)dt \notag\\
&=&\frac{1}{\sqrt{2\pi}}\exp\left(\frac{\sigma^2}{2}T\right)\int\limits_{\chi}^{\infty}\exp\left(-\frac{(\sigma\sqrt{T}-z)^2}{2}\right)dz \notag\\
&=&\frac{1}{\sqrt{2\pi}}\int\limits_{\chi}^{\infty}\exp\left(-\frac{z^2}{2}+z\sigma\sqrt{T}\right)dz. \notag
\end{eqnarray}
We have
\begin{eqnarray}
\Psi&=&\frac{1}{\sqrt{2\pi}}S(0)\exp\left((\mu-\frac{\sigma^2}{2})\cdot T\right)\int_{\chi}^{\infty}\exp\left(-\frac{s^2}{2}+\sigma\sqrt{T}s\right)ds \notag\\
&=&S(0)\exp(\mu T)\exp\left(-\frac{\sigma^2}{2}T\right)\frac{1}{\sqrt{2\pi}}\int_{\chi}^{\infty}\exp\left(-\frac{s^2}{2}+\sigma\sqrt{T}s\right)ds \notag\\
&=&S(0)\exp(\mu T)\frac{1}{\sqrt{2\pi}}\exp\left(-\frac{\sigma^2}{2}T\right)\exp\left(\frac{\sigma^2}{2}T\right)\int\limits_{-\infty}^{\sigma\sqrt{T}-\chi}\exp\left(-\frac{s^2}{2}\right)ds \notag\\
&=&S(0)\exp(\mu T)\Phi(\sigma\sqrt{T}-\chi). \notag
\end{eqnarray}

\section*{Task 4}

\section*{Task 5}
We have $\Phi^{-1}(0)=-\infty$ and $\Phi^{-1}(1)=\infty$ where $\Phi^{-1}:(0,1)\to(-\infty,\infty)$ is the inverse cumulative distribution function. As an integral of a positive continious function, $\Phi$ is a bijection, continious and differentiable. That means $\Phi$ is a diffeomorphism. $\Phi^{-1}$ is also differentiable, so we can use the transformation theorem with the change-of-variable $t=\Phi(s)$. We have $|\det(D\Phi(s))|=\frac{1}{\sqrt{2\pi}}\exp(-\frac{s^2}{2})$ and
\begin{eqnarray}
&&\int_0^1 f(\Phi^{-1}(t))dt \notag\\
&=&\int_{\Phi^{-1}(0)}^{\Phi^{-1}(1)}f(\Phi^{-1}(\Phi(s)))\cdot|\det(D\Phi(s))|ds \notag\\
&=&\int_{-\infty}^{\infty}f(s)\frac{1}{\sqrt{2\pi}}\exp\left(-\frac{s^2}{2}\right)ds \notag\\
&=&\frac{1}{\sqrt{2\pi}}\int_{-\infty}^{\infty}f(s)\exp\left(-\frac{s^2}{2}\right)ds \notag
\end{eqnarray}
what proves formula (7).

\section*{Task 6}

\section*{Task 7}

\section*{Task 8}

\section*{Task 9}

\section*{Task 10}


\end{document}